% For faster processing, load Matlab syntax for listings
\definecolor{MyDarkGreen}{rgb}{0.0,0.4,0.0}
\lstloadlanguages{Matlab}%
\lstset{language=Matlab,
        frame=single,
        basicstyle=\small\ttfamily,
        keywordstyle=[1]\color{Blue}\bf,
        keywordstyle=[2]\color{Purple},
        keywordstyle=[3]\color{Blue}\underbar,
        identifierstyle=,
        commentstyle=\usefont{T1}{pcr}{m}{sl}\color{MyDarkGreen}\small,
        stringstyle=\color{Purple},
        showstringspaces=false,
        tabsize=5,
        % Put standard MATLAB functions not included in the default
        % language here
        morekeywords={xlim,ylim,var,alpha,factorial,poissrnd,normpdf,normcdf},
        % Put MATLAB function parameters here
        morekeywords=[2]{on, off, interp},
        % Put user defined functions here
        morekeywords=[3]{FindESS},
        morecomment=[l][\color{Blue}]{\ldots},
        numbers=left,
        firstnumber=1,
        numberstyle=\tiny\color{Blue},
        stepnumber=0
}

% Only the next five fields need to be edited.
\newcommand{\lecAuth}{R.A. de Callafon}
\newcommand{\scribe}{Thomas Denewiler}
\newcommand{\authEmail}{callafon@ucsd.edu}
\newcommand{\scribeEmail}{tdenewiler@gmail.com}
\newcommand{\course}{MAE 283: Parameter Estimation}

\address{Department of Mechanical and Aerospace Engineering, University of California, San Diego}

% Adds a hyperlink to an email address.
\newcommand{\mailto}[2]{\href{mailto:#1}{#2}}

% These commands set the document properties for the PDF output. Needs the hyperref package.
\hypersetup{%
    colorlinks,
    linkcolor={black},
    citecolor={black},
    filecolor={black},
    urlcolor={black},
    pdfauthor={\scribe<\mailto{\scribeEmail}{\scribeEmail}>},
    pdfsubject={\course},
    pdftitle={MAE 283 Lecture Notes},
    pdfkeywords={UC San Diego, Parameter Estimation, System Identification},
    pdfstartpage={1},
}

% Includes a figure
% The first parameter is the label, which is also the name of the figure
%   with or without the extension (e.g., .eps, .fig, .png, .gif, etc.)
%   IF NO EXTENSION IS GIVEN, LaTeX will look for the most appropriate one.
%   This means that if a DVI (or PS) is being produced, it will look for
%   an eps. If a PDF is being produced, it will look for nearly anything
%   else (gif, jpg, png, et cetera). Because of this, when I generate figures
%   I typically generate an eps and a png to allow me the most flexibility
%   when rendering my document.
% The second parameter is the width of the figure normalized to column width
%   (e.g. 0.5 for half a column, 0.75 for 75% of the column)
% The third parameter is the caption.
\newcommand{\scalefig}[3]{%
  \begin{figure}[ht!]
    % Requires \usepackage{graphicx}
    \centering
  \fbox{%
      \includegraphics[width=#2\columnwidth]{#1}
  }
    %%% I think \captionwidth (see above) can go away as long as
    %%% \centering is above
    %\captionwidth{#2\columnwidth}%
    \caption{#3}
\label{#1} % NOLINT
  \end{figure}}

% Includes a MATLAB script.
% The first parameter is the label, which also is the name of the script
%   without the .m.
% The second parameter is the optional caption.
\newcommand{\matlabscript}[2]
  {\begin{itemize}\item[]\lstinputlisting[caption=#2,label=#1]{#1.m}\end{itemize}}

% Example environment.
\newtheoremstyle{example}{\topsep}{\topsep} %
     {}%         Body font
     {}%         Indent amount (empty = no indent, \parindent = para indent)
     {\bfseries}% Thm head font
     {}%        Punctuation after thm head
     {\newline}%     Space after thm head (\newline = linebreak)
     {\thmname{#1}\thmnumber{#2}\thmnote{#3}}%Thm head spec

   \theoremstyle{example}
   \newtheorem{example}{Example}[section]

% A command to show a vector norm that will have the pipe signs scale with the contents.
\newcommand{\vectornorm}[1]{\left|\left|#1\right|\right|}
\newcommand{\argmin}[1]{\underset{#1}{\operatorname{argmin}}}
\newcommand{\argmax}[1]{\underset{#1}{\operatorname{argmax}}}

% Commands for time and frequency integrals over infinty, cos and sin.
\newcommand{\tint}{\int_{t=-\infty}^\infty}
\newcommand{\fint}{\int_{\omega=-\infty}^\infty}
\newcommand{\tauint}{\int_{\tau=0}^\infty}
\newcommand{\tausum}{\sum_{\tau=0}^\infty}
\newcommand{\w}{\omega}
\newcommand{\wo}{\omega_0}
\newcommand{\ejwt}{e^{j\omega\ t}}
\newcommand{\emjwt}{e^{-j\omega\ t}}
\newcommand{\dt}{\Delta_T}
\newcommand{\vp}{\varphi}
\newcommand{\fN}{\frac{1}{N}}
\newcommand{\sumt}{\sum_{t=1}^N}
\newcommand{\sumk}{\sum_{k=0}^N}
\newcommand{\ruhat}{\hat{R}_u^N (\tau)}
\newcommand{\ryuhat}{\hat{R}_{yu}^N (\tau)}
\newcommand{\phiuhat}{\hat{\Phi}_u^N (\omega)}
\newcommand{\phiyuhat}{\hat{\Phi}_{yu}^N (\omega)}
\newcommand{\phiyuhh}{\hat{\hat{\Phi}}_{yu}^N (\omega)}
\newcommand{\phiuhh}{\hat{\hat{\Phi}}_u^N (\omega)}
\newcommand{\phiyhh}{\hat{\hat{\Phi}}_y^N (\omega)}
\newcommand{\ghh}{\hat{\hat{G}} (e^{j\omega})}
\newcommand{\thn}{\hat{\theta}_{LS}^N}
\newcommand{\thiv}{\hat{\theta}_{IV}^N}
\newcommand{\thetan}{\hat{\theta}^N}
\newcommand{\btheta}{\bar{\theta}}
\newcommand{\tast}{\theta^\ast}
\newcommand{\ejw}{(e^{j\omega})}
\newcommand{\onen}{\frac{1}{N}}
\newcommand{\tonen}{\tfrac{1}{N}}
\newcommand{\ptheta}{\frac{\partial}{\partial\theta}}
\newcommand{\solth}{\text{sol}_\theta}
\newcommand{\cov}{\text{cov}}
\newcommand{\var}{\text{var}}
\newcommand{\T}{\theta}
\newcommand{\eps}{\epsilon}
\newcommand{\ett}{\epsilon(t,\theta)}
\newcommand{\est}{\epsilon^2(t,\theta)}
\newcommand{\go}{G_0(q)}
\newcommand{\gt}{G_\theta(q)}
\newcommand{\ho}{H_0(q)}
\newcommand{\hth}{H_\theta(q)}
\newcommand{\hit}{H_\theta^{-1} (q)}
\def\argmin{\mathop{\arg\,\min}\limits}
\def\argmax{\mathop{\arg\,\max}\limits}
\def\argsol{\mathop{\arg\,\text{sol}}\limits}
\def\sign{\mathop{\text{sign}}\limits}
\def\ent{\mathop{\text{ent}}\limits}
